
\documentclass[12pt,a4paper]{article}

\usepackage{amsmath,amssymb}
\usepackage[utf8]{inputenc}                                      
\usepackage[OT4]{fontenc}      
%\usepackage[T1]{fontenc}                            
\usepackage[polish]{babel}                           
\selectlanguage{polish}
\usepackage{indentfirst} 
\usepackage[dvips]{graphicx}
\usepackage{tabularx}
\usepackage{color}
\usepackage{hyperref} 
\usepackage{fancyhdr}
\usepackage{listings}
\usepackage{booktabs}
\usepackage{ifpdf}
\usepackage{mathtext} % polskie znaki w trybie matematycznym
%\makeindex  % utworzenie skorowidza (w dokumencie pdf)
\usepackage{lmodern}
%\usepackage[osf]{libertine}
\usepackage{filecontents}


\usepackage{tikz}
\usetikzlibrary{arrows}


\newcounter{nextYear}
\setcounter{nextYear}{\the\year}
\stepcounter{nextYear}

% rozszerzenie nieco strony
%\setlength{\topmargin}{-1cm} \setlength{\textheight}{24.5cm}
%\setlength{\textwidth}{17cm} \addtolength{\hoffset}{-1.5cm}
%\setlength{\parindent}{0.5cm} \setlength{\footskip}{2cm}
%\linespread{1.2} % odstep pomiedzy wierszami

\ifpdf
\DeclareGraphicsRule{*}{mps}{*}{}
\fi


\newcommand{\tl}[1]{\textbf{#1}} 
\pagestyle{fancy}
\renewcommand{\sectionmark}[1]{\markright{\thesection\ #1}}
\fancyhf{} % usuwanie bieżących ustawień
\fancyhead[LE,RO]{\small\bfseries\thepage}
\fancyhead[LO]{\small\bfseries\rightmark}
\fancyhead[RE]{\small\bfseries\leftmark}
\renewcommand{\headrulewidth}{0.5pt}
\renewcommand{\footrulewidth}{0pt}
\addtolength{\headheight}{0.5pt} % pionowy odstęp na kreskę
\fancypagestyle{plain}{%
\fancyhead{} % usuń p. górne na stronach pozbawionych numeracji
\renewcommand{\headrulewidth}{0pt} % pozioma kreska
}

% ustawienia listingu programow
%\lstset{	language=C++, 
%        	numbers=left, 
%        	numberstyle=\tiny, 
%        	stepnumber=1, 
%        	numbersep=5pt,
%		  	stringstyle=\ttfamily,
%			showstringspaces=false,
% 			tabsize=4
%		}

\lstset{%
language=C++,%
commentstyle=\textit,%
identifierstyle=\textsf,%
keywordstyle=\sffamily\bfseries, %
%captionpos=b,%
tabsize=3,%
frame=lines,%
numbers=left,%
numberstyle=\tiny,%
numbersep=5pt,%
breaklines=true,%
morekeywords={pWezel,Wezel,string,ref,params_result,time_t},%
escapeinside={(*@}{@*)},%
%basicstyle=\footnotesize,%
%keywords={double,int,for,if,return,vector,matrix,void,public,class,string,%
%float,sizeof,char,FILE,while,do,const}
}
%%%%%%%%%%%%%%%%%%%%%%%%%%%%%%%%%%%%%%%%%%%%%%%%%%%%%%%%%%%%%%%%%%%%%%%

% mala zmiana sposobu wyswietlania notek bocznych
\let\oldmarginpar\marginpar
\renewcommand\marginpar[1]{%
  {\linespread{0.85}\normalfont\scriptsize%
%   \oldmarginpar[\vspace{-1.5ex}\raggedleft\scriptsize\color{black}\textsf{#1}]%    left pages
%                {\vspace{-1.5ex}\raggedright\scriptsize\color{black}\textsf{#1}}% right pades
\oldmarginpar[\hspace{1cm}\begin{minipage}{3cm}\raggedleft\scriptsize\color{black}\textsf{#1}\end{minipage}]%    left pages
{\hspace{0cm}\begin{minipage}{3cm}\raggedright\scriptsize\color{black}\textsf{#1}\end{minipage}}% right pages
}%
}
% % % % % % % % % % % % % % % % % % % % % % % % % % % % % % % %


\begin{document}
\frenchspacing
\thispagestyle{empty}
\begin{center}
{\Large\sf Politechnika Śląska w Gliwicach   % Alma Mater

Wydział Automatyki, Elektroniki i Informatyki

}

\vfill

%\includegraphics[width=0.15\textwidth]{graf/polsl.pdf}

\vfill\vfill

{\Huge\sffamily\bfseries Podstawy Programowania Komputerów} \\ % tu podać nazwę przedmiotu

\vfill\vfill

{\LARGE\sf Słońce}  % a tu temat laborki :-)


\vfill \vfill\vfill\vfill

%%%%%%%%%%%%%%%%%%%%%%%%%%%%


\begin{tabular}{ll}
\toprule
	autor                                                  & Tomasz Sojka          \\
	prowadzący                                             & dr inż. Krzysztof Simiński       \\
	rok akademicki                                         & \the\year/\arabic{nextYear} \\
	kierunek                                               & informatyka                 \\
	rodzaj studiów                                         & SSI                         \\
	semestr                                                & 1                           \\
	termin laboratorium / ćwiczeń                          & poniedziałek, 08:30 -- 10:00      \\
	grupa                                                  & 2                           \\
	sekcja                                                 & 6                           \\
	termin oddania sprawozdania                            & \the\year-12-01             \\
	data oddania sprawozdania                              & \the\year-12-01             \\
\bottomrule &  \\
\end{tabular}

\end{center}
%%%%%%%%%%%%%
%%%%%%%%%%%%%%%%%%%%%%%%%%%%%%%%%%%%%%%%%%%%%%%%%%%%%%%%%%%%%%%%%%%%%%%%%
\cleardoublepage
%%%%%%%%%%%%%%%%%%%%%%%%%%%%%%%%%%%%%%%%%%%%%%%%%%%%%%%%%%%%%%%%%%%%%%%%%

%%%%%%%%%%%%%%%%%%%%%%%%%%%%%%%%%%%%%%%%%%%%%%%%%%%%%%%%%%%%%%%%%%%%%%%%%
\section{Treść zadania}
Napisac program wyznaczajacy godzine wschodu i zachodu Słonca dla podanych dni i podanej lokalizacji
na Ziemi.
Program uruchamiany jest z linii poleceń z wykorzystaniem następujących przełączników:
 
\begin{tabular}{ll}
\texttt{-o}  & nazwa pliku wyjściowego\\
\texttt{-s}  & data poczatkowa w formacie rrrr-mm-dd, np. 2017-10-14\\
\texttt{-k}  & data koncowa w formacie rrrr-mm-dd\\
\texttt{--lon}  & długosc geograficzna w stopniach\\
\texttt{} &(dodatnia na wschodzie, ujemna na zachodzie)\\
\texttt{--lat}  & szerokosc geograficzna w stopniach \\
\texttt{} &(dodania na północy, ujemna na południu)\\
\texttt{-t}  & strefa czasowa\\
\end{tabular}

%%%%%%%%%%%%%%%%%%%%%%%%%%%%%%%%%%%%%%%%%%%%%%%%%%%%%%%%%%%%%%%%%%%%%%%%%
\section{Analiza zadania}


Zagadnienie przedstawia problem obliczania godzin wschodów i zachodów słońca, dla podanych współrzędnych geograficznych,
od podanej daty poczatkowej, do końcowej.

\subsection{Struktury danych}
W programie wykorzystano struktury i typy danych z biblioteki \texttt{time.h} (patrz \ref{sec:sp-wew}). Dane przypisane do struktury tm, otrzymały zakres ??.  
Zamiana struktury na typ danych \lstinline|time_t| za pomocą funkcji mktime, pozowliła na korekcje danych spoza zakresu na poprawną datę.
Taka operacja przyspieszyła proces tworzenia programu i ułatwiła dane zadanie.


\subsection{Algorytmy}
W programie użyty został algorytm, który pozwalał na wyliczenie godziny wschodu i zachodu słońca za pomocą astronomicznych danych ziemi. Algorytm korzystał z podanej szerokości, długości geograficznej i podanej daty. Poprzez umieszczenie algorytmu  w pętli (patrz \ref{sec:sp-wew}), program oblicza te godziny od daty początkowej do końcowej . 

 
%%%%%%%%%%%%%%%%%%%%%%%%%%%%%%%%%%%%%%%%%%%%%%%%%%%%%%%%%%%%%%%%%%%%%%%%%
\section{Specyfikacja zewnętrzna}
\label{sec:sp:zewnetrzna}
Program jest uruchamiany z linii poleceń. 
Należy przekazać do programu, po odpowiednich przełącznikach: nazwę pliku wyjściowego, datę początkową, datę końcową, strefę, szerokość i długość geograficzną  (odpowiednio: \texttt{-o} dla pliku wyjściowego , \texttt{-s} dla daty początkowej , \texttt{-k} dla daty końcowej , \texttt{-t} dla strefy , \texttt{--lat} dla szerokości geograficznej , \texttt{--lon} dla długości geograficznej ), np.
\begin{verbatim}
program.exe -o Gliwice -s 2017-12-01 -k 2017-12-31 -t 1 
--lat 50.310 --lon 18.669
\end{verbatim}
Pliki są plikami tekstowymi, ale mogą mieć dowolne rozszerzenie (lub go nie mieć.) Przełączniki mogą być podane w dowolnej kolejności. Uruchomienie programu z nieprawidłowymi parametrami lub podanie danych spoza zakresu powoduje wyświetlenie komunikatu 
\begin{verbatim}
Nieprawidlowe parametry lub dane spoza zakresu
\end{verbatim}
i wyświetlenie pomocy. 



%%%%%%%%%%%%%%%%%%%%%%%%%%%%%%%%%%%%%%%%%%%%%%%%%%%%%%%%%%%%%%%%%%%%%%%%%
\section{Specyfikacja wewnętrzna}\label{sec:sp-wew}
 Program został zrealizowany zgodnie z paradygmatem strukturalnym.  
W programie rozdzielono interfejs (komunikację z użytkownikiem) od logiki aplikacji (obliczania godzniy wschodu i zachodu słońca, oraz wpisywania danych do pliku).

\subsection{Ogólna struktura programu}
W funkcji głównej wywołana jest instrukcja warunkowa if, sprawdzająca czy podane współrzędne geograficzne lub strefa lub rok początkowy lub rok końcowy nie należą do ich zakresu, oraz czy funkcja 
\begin{lstlisting}
bool odczytajargumenty(int ile, char * argumenty[], string & nazwa, double & Lat, double & Lon, int & R, int & M, int & D, int & R2, int & M2, int & D2, int & Strefa);
\end{lstlisting}
przyjmuje wartość \lstinline!false! (funkcja \lstinline|odczytajargumenty| odczytuje przełączniki programu i sprawdza czy zostały one wprowadzone w prawidłowy sposób). Jeśli instrukcja warunkowa jest spełniona zostaje wyświetlony odpowiedzni komunikat i program kończy się.

Po podaniu prawidłowych parametrów i danych wywoływana jest funkcja 
\begin{lstlisting}
funkcjakoncowa(int &R, int &M, int &D, int &R2, int &M2, int &D2, int &Stref, int &sekundy, double &Lat, double &Lon, string &nazwa);
\end{lstlisting} 
W tej funkcji zawarta jest cała konstrukcja programu. \\
Funcja rozpoczyna się od operacji na podanej dacie początkowej. Zadeklarowany jest wskaźnik \lstinline!data! na strukturę tm i zmienna \lstinline!datasek! typu time\_t (czas w sekundach od 01-01-1970, 00:00). Do zmiennej datasek, za pomocą funkcji time, zapisywany jest lokalny czas, który zostaje przekonwertowany i przypisany wskaźnikowi \lstinline!data!. 
W następnym  kroku wywołane są funkcje:
\begin{lstlisting}
wprowadzaniedanych(int R, int M, int D, tm*data);
time_t timetdata(int R, int M, int D, tm*data, time_t datasek);
\end{lstlisting}
Pierwsza z nich służy do przypisania daty początkowej do wskaźnika \lstinline!data!, druga za pomocą funkcji mktime zamienia wcześniej podaną datę początkową na czas typu time\_t i przypisuje go zmiennej \lstinline!datasek!.
\\* 
\\* 
Cała operacja powtarza się dla daty końcowej (wskaźnik na strukture tm --  \lstinline!data2!, zmienna typu time\_t -  \lstinline!datasek2!).
Następnie za pomocą funkcji difftime liczona jest różnica \lstinline!sekundy! dni od daty początkowej do końcowej (włącznie z datą początkową i końcową). 
 \\* 
\\*
Wywołana jest funkcja:
\begin{lstlisting}
wypisywanienaglowka(const string &nazwa, double Lat, double Lon, int Stref);
\end{lstlisting}
wypisująca do pliku nagłówek z napisem WSCHODY I ZACHODY SLONCA, oraz danymi: szerokością, długością i strefą.Wypisywane są równeż tytuły kolumn z danymi (data, wschod, zachod).
 \\* 
\\*
Następnym krokiem jest wypisanie za pomocą funkcji:
\begin{lstlisting}
wypisywaniegodziny(const string &nazwa, time_t datasek, tm*data, double sekundy, int R, int M, int D, double Lat, double Lon, double Stref);
\end{lstlisting}
godzin wschodów i zachodów słońca dla kolejnych dni. Funkcja ta zawiera w pętli for dwie funkcje z algorytmem obliczającym godzinę wschodu i zachodu:
\begin{lstlisting}
double obliczeniawschodu(int R, int M, int D, double Lat, double Lon, int Stref);
double obliczeniazachodu(int R, int M, int D, double Lat, double Lon, int Stref);
\end{lstlisting}
oraz wypisuje do pliku te godziny.   
 Wynik funkcji difftime (\lstinline!sekundy!) jest równy ilości wywołań pętli for. 

\section{Testowanie}
Program został przetestowany na różnego rodzaju danych. Dni lub miesiące daty początkowej i końcowej mogą nie należeć do zakresu (np. 2000-125-562), są wtedy zamieniane na poprawną datę. Szerokość, długość lub strefa spoza zakresu powodują zgłoszenie błędu. Podanie roku wcześniejszego niż 1970(od wtedy jest liczony czas time\_t wykorzystany w programe) powoduje zgłoszenie błedu. Podanie wcześniejszej daty końcowej od poczatkowej powoduje wypisanie tylko nagłówka. Nie podanie parametru lub podanie błędnego (np. data abcd-de-fg) powoduje zgłoszenie błędu. 


\section{Wnioski}


Program Słońce, choć wydaje się być prostym programem, początkującemu programiście sprawi wiele problemów. Program wymaga zastosowania wielu instrukcji warunkowych, aby program działał poprawnie dla każdych współrzędnych i każdego dnia. Największymi problemami podczas wykonania zadania były operacje na bibliotece time.h,  tworzenie funkcji odczytującej parametry i ograniczenie czasowe, oraz konieczność nauki innych przedmiotów.

Program dla większości danych generuje się poprawnie, lecz brak w nim instrukcji warunkowej związanej z nocami i dniami polarnymi.


\end{document}
% Koniec wieńczy dzieło.
