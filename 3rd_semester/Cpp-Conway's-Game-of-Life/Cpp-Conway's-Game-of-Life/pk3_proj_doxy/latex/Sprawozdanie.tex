
\documentclass[12pt,a4paper]{article}

\usepackage{amsmath,amssymb}
\usepackage[utf8]{inputenc}                                      
\usepackage[OT4]{fontenc}      
%\usepackage[T1]{fontenc}                            
\usepackage[polish]{babel}                           
\selectlanguage{polish}
\usepackage{indentfirst} 
\usepackage[dvips]{graphicx}
\usepackage{tabularx}
\usepackage{color}
\usepackage{hyperref} 
\usepackage{fancyhdr}
\usepackage{listings}
\usepackage{booktabs}
\usepackage{ifpdf}
\usepackage{mathtext} % polskie znaki w trybie matematycznym
%\makeindex  % utworzenie skorowidza (w dokumencie pdf)
\usepackage{lmodern}
%\usepackage[osf]{libertine}
\usepackage{filecontents}


\usepackage{hyperref}

\usepackage{tikz}
\usetikzlibrary{arrows}


\newcounter{nextYear}
\setcounter{nextYear}{\the\year}
\stepcounter{nextYear}

% rozszerzenie nieco strony
%\setlength{\topmargin}{-1cm} \setlength{\textheight}{24.5cm}
%\setlength{\textwidth}{17cm} \addtolength{\hoffset}{-1.5cm}
%\setlength{\parindent}{0.5cm} \setlength{\footskip}{2cm}
%\linespread{1.2} % odstep pomiedzy wierszami

\ifpdf
\DeclareGraphicsRule{*}{mps}{*}{}
\fi


\newcommand{\tl}[1]{\textbf{#1}} 
\pagestyle{fancy}
\renewcommand{\sectionmark}[1]{\markright{\thesection\ #1}}
\fancyhf{} % usuwanie bieżących ustawień
\fancyhead[LE,RO]{\small\bfseries\thepage}
\fancyhead[LO]{\small\bfseries\rightmark}
\fancyhead[RE]{\small\bfseries\leftmark}
\renewcommand{\headrulewidth}{0.5pt}
\renewcommand{\footrulewidth}{0pt}
\addtolength{\headheight}{0.5pt} % pionowy odstęp na kreskę
\fancypagestyle{plain}{%
\fancyhead{} % usuń p. górne na stronach pozbawionych numeracji
\renewcommand{\headrulewidth}{0pt} % pozioma kreska
}

% ustawienia listingu programow
%\lstset{	language=C++, 
%        	numbers=left, 
%        	numberstyle=\tiny, 
%        	stepnumber=1, 
%        	numbersep=5pt,
%		  	stringstyle=\ttfamily,
%			showstringspaces=false,
% 			tabsize=4
%		}

\lstset{%
language=C++,%
commentstyle=\textit,%
identifierstyle=\textsf,%
keywordstyle=\sffamily\bfseries, %
%captionpos=b,%
tabsize=3,%
frame=lines,%
numbers=left,%
numberstyle=\tiny,%
numbersep=5pt,%
breaklines=true,%
morekeywords={pWezel,Wezel,string,ref,params_result,time_t},%
escapeinside={(*@}{@*)},%
%basicstyle=\footnotesize,%
%keywords={double,int,for,if,return,vector,matrix,void,public,class,string,%
%float,sizeof,char,FILE,while,do,const}
}
%%%%%%%%%%%%%%%%%%%%%%%%%%%%%%%%%%%%%%%%%%%%%%%%%%%%%%%%%%%%%%%%%%%%%%%

% mala zmiana sposobu wyswietlania notek bocznych
\let\oldmarginpar\marginpar
\renewcommand\marginpar[1]{%
  {\linespread{0.85}\normalfont\scriptsize%
%   \oldmarginpar[\vspace{-1.5ex}\raggedleft\scriptsize\color{black}\textsf{#1}]%    left pages
%                {\vspace{-1.5ex}\raggedright\scriptsize\color{black}\textsf{#1}}% right pades
\oldmarginpar[\hspace{1cm}\begin{minipage}{3cm}\raggedleft\scriptsize\color{black}\textsf{#1}\end{minipage}]%    left pages
{\hspace{0cm}\begin{minipage}{3cm}\raggedright\scriptsize\color{black}\textsf{#1}\end{minipage}}% right pages
}%
}
% % % % % % % % % % % % % % % % % % % % % % % % % % % % % % % %


\begin{document}
\frenchspacing
\thispagestyle{empty}
\begin{center}
{\Large\sf Politechnika Śląska w Gliwicach   % Alma Mater

Wydział Automatyki, Elektroniki i Informatyki

}

\vfill

%\includegraphics[width=0.15\textwidth]{graf/polsl.pdf}

\vfill\vfill

{\Huge\sffamily\bfseries Programowanie Komputerów 3} \\ % tu podać nazwę przedmiotu

\vfill\vfill

{\LARGE\sf GRA W ŻYCIE}  % a tu temat laborki :-)


\vfill \vfill\vfill\vfill

%%%%%%%%%%%%%%%%%%%%%%%%%%%%


\begin{tabular}{ll}
\toprule
	autor                                                  & Tomasz Sojka          \\
	prowadzący                                             & mg inż. Grzegorz Kwiatkowski       \\
	rok akademicki                                         & 2018/\the\year \\
	kierunek                                               & informatyka                 \\
	rodzaj studiów                                         & SSI                         \\
	semestr                                                & 3                           \\
	termin laboratorium / ćwiczeń                          & wtorek, 13:45 -- 15:15      \\
	grupa                                                  & 2                           \\
\bottomrule &  \\
\end{tabular}

\end{center}
%%%%%%%%%%%%%
%%%%%%%%%%%%%%%%%%%%%%%%%%%%%%%%%%%%%%%%%%%%%%%%%%%%%%%%%%%%%%%%%%%%%%%%%
\cleardoublepage
%%%%%%%%%%%%%%%%%%%%%%%%%%%%%%%%%%%%%%%%%%%%%%%%%%%%%%%%%%%%%%%%%%%%%%%%%

\section{Informacja}
\subsection{Diagram UML}
 Pełen diagram UML mozna znalezc pod linkiem: \\
\url{https://go.gliffy.com/go/share/sbejb2lxs2s3y1t601g8}
\subsection{Kod i pliki testowe}
Pełny kod programu oraz pliki tekstowe do testowania znajdują się na moim repozytorium: \\
\url{https://github.com/tomaszsojka/my/tree/master/projekt_pk3}
\subsection{Specyfikacja wewnętrzna }

Poniżej znajduje się dokumentacja klas i plików. \\
Między innymi opis klas i ich metod, opis funkcji oraz operatorów, grafy wywołania, diagramy dziedziczenia i współpracy klas. \\
Zaawansowany c++:
W programie użyto wątków w celu synchronizacji dwóch metod klasy menue. Ich opis można znaleźć w dokumentacji metod klasy menue:  \textbf{petla} i \textbf{zatrzymaj\_petle}, oraz w pełnym kodzie w pliku metody\_menue.cpp w metodzie \textbf{poprowadz\_uzytkownika}.
\label{id:sec:specyfikacja}
\end{document}
% Koniec wieńczy dzieło.
